% Compiled with:
% env TEXMFDOTDIR='..//:~/src/texlive/texmf-dist/source/latex/media9/files/' lualatex pdfextra-example-latex.tex
\documentclass{article}

\usepackage{hyperref}
\usepackage{xcolor}
\usepackage{graphicx}
\usepackage{pdfextra}
\sdef{_linkcolor}{\color{blue}}

\begin{document}

% Using document level JavaScript we can run initialization code right at the
% start (this is more flexible than \openaction).
\dljavascript[random]{
  function getRandomNumber() {
      return 4; % chosen by fair dice roll, ![https://xkcd.com/221/]
  }
  console.println("initialized with seed " + getRandomNumber());
}



% For external references we need to `\filedef` "external" and "url" files
% beforehand:
\filedef/u[doc-internet]{http://petr.olsak.net/ftp/olsak/optex/optex-doc.pdf}
\filedef/x[doc-local]{optex-doc.pdf}

% We then try to reference them using external references:
\hlink[extref:doc-internet:ref:langphrases]{OpTeX documentation, section ``Multilingual phrases and quotation marks''.}
\hlink[extpgref:doc-internet:12]{OpTeX documentation, page 12.}



% Insert a video twice with different settings:
\filedef/e[video]{example-movie.mp4}
% Insert video into page using Renditions mechanism with controls and autoplay.
\render[video][
  controls=true,
  aactions=\renditionautoplay,
]{\includegraphics[width=\textwidth]{example-image.pdf}}

% We set red text color, blue border (RGB triplet) for "rendition" links.
\sdef{_renditionlinkcolor}{\color{red}}
\sdef{_renditionborder}{0 1 0}
% Then we add some buttons.
\hlink[rendition::play]{play} \hlink[rendition::pause]{pause}
\hlink[rendition::stop]{stop} \hlink[rendition::resume]{resume}

% Render the same file again, but with different dimensions, no controls
% and explicit activation.
\render[video]{\includegraphics{example-image.pdf}}
% This one can't have buttons, because it doesn't have its own name -- "video"
% is used by the previous one.



% Add already embedded video to the list of attachments shown in user
% interface.
\attach[video]


% We are done for this page, who not go to the next one?
%
% \hlink typesets in group, so we locally disable lining links (our button is
% bigger than normal line)
\hlink[named:NextPage]{\nolininglinks\scalebox{2}[2]{NEXT PAGE}}



\clearpage



% Deliver fresh random number to everyone who sees this page using "page open"
% ("O") trigger event of page's actions.
\pageactions[
  {O} {js:{app.alert("Random number for you is " + getRandomNumber());}}
]


% Link to previous page (predefined PDF action) with "Split" transition:
\hlink[named:PrevPage, transition:Split]{Go to previous page (with transition).}



\clearpage

% A 3D Rich Media annotation with a custom view and script follows:

% First we define the 3D file, so we can use same name for the view and have it
% auto included.
\filedef/e[example]{pdfextra-example-part.prc}

% Then a simplistic "front" view.
\DDDview[example][
  projection=ortho,
  roo=400,
]

% A 3D JavaScript script that will define function to use with action button.
% (we write the script only now, so we don't have to carry around an extra file
% in the package, you would normally have the contents in the file -- "turn.js"
% in this case).
\newwrite\js
\immediate\openout\js=turn.js
% rotate every node by angle degrees along $x$ axis
\immediate\write\js{\detokenize{
function turn(angle) {
    angle = angle * Math.PI / 180;
    for (var i = 0; i < scene.nodes.count; i++) {
        var node = scene.nodes.getByIndex(i);
	node.transform.rotateAboutXInPlace(angle);
    }
}
}}
\closeout\js

% The annotation itself, with our script above.
\RM[example][
  activation=auto,
  scripts=turn.js,
]{\textcolor{white}{\vrule width\textwidth height\textwidth}}

\hlink[js:\DDDcontext{example}.turn(10);]{Turn by 10 degrees along $x$ axis.}\par
\hlink[js:\DDDcontext{example}.turn(30);]{Turn by 30 degrees along $x$ axis.}\par
\hlink[js:\DDDcontext{example}.turn(90);]{Turn by 90 degrees along $x$ axis.}



\clearpage


% Here is the "epix" 3D model example from media9 documentation ported to
% pdfextra.
%
% First we define a view with the same name as the future Rich Media
% annotation. This view will then automatically be used if no `views` key is
% specified. No need to be excessive with decimal numbers.
\DDDview[epix][
  FOV=30,
  c2c=4 2 3,
  coo=4.41 2.19 0,
  roo=430,
]

% 3Dmenu is used often, lets use shortcut instead.
\filedef/e[3dmenu]{3Dmenu.js}
% We want to use the nice "epix" name for the view and annotation.
\filedef/e[epix]{epix.prc}

% Just to spice things up, we diverge a bit from media9 and define custom
% background color. We can't use LaTeX's colors, so we use RGB triplet.
\newcommand{\GreyBG}{0.9 0.9 0.9}

% Define the 3D Rich Media annotation. One view is automatically detected from
% name ("epix"). Scripts will run in specified order at initialization time
% (when the annotation is activated).
\RM[epix][
  scripts={
    asylabels.js,
    3Dspintool.js,
    3dmenu,
  },
  background=\GreyBG,
]{\includegraphics[width=0.8\textwidth, height=0.8\textwidth]{epixposter.pdf}}



\clearpage



% Second media9 example is "dice" it has a lot of views, but they are simple.
% This time we can't use the view name deduction from `\RM` name -- there is
% more than one view.
\DDDview[Front][roo=27]
\DDDview[Back][roo=27, c2c=0 1 0]
\DDDview[Left][roo=27, c2c=-1 0 0]
\DDDview[Right][roo=27, c2c=1 0 0]
\DDDview[Top][roo=27, c2c=0 0 1]
\DDDview[Bottom][roo=27, c2c=0 0 -1]
\DDDview[Fish Eye at Centre][fov=120]

% We want to use all the defined views, as well as autoactivate the annotation.
% This means that the preview mostly sets dimensions, because it will not be
% shown much. (Only in viewers that don't support 3D Rich Media -- that is, all
% except Acrobat).
\RM[dice.u3d][
  views={Front, Back, Left, Right, Top, Bottom, Fish Eye at Centre},
  activation=auto,
  scripts=3dmenu,
]{\textcolor{red}{\vrule height.5\textwidth width.5\textwidth}}

% A few buttons for switching views.
\hlink[goto3dview::/N]{\frame{Next view}}
\hlink[goto3dview::(Back)]{\frame{View ``Back''}}
\hlink[goto3dview::5]{\frame{6th view in the list}}

\clearpage


% Last media9 example ported. It has its own JavaScript file for animation,
% which can be customized at runtime using JavaScript actions that access 3D
% context of the annotation.
\filedef/e[malte]{malte.u3d}
\DDDview[malte][c2c=1 1 1, roo=0.25]
\RM[malte][
  activation=auto,
  scripts={3dmenu, malte.js},
]{\textcolor{blue}{\vrule height 0.5\textwidth width 0.5\textwidth}}

% Normal JavaScript actions, but we access functions defined globaly in the 3D
% JavaScript context of "malte" Rich Media annotation.
\hlink[js:\DDDcontext{malte}.cntrClockWise();]  {\includegraphics[height=1.44em]{boutona.pdf}}
\hlink[js:\DDDcontext{malte}.pause();]          {\includegraphics[height=1.44em]{boutonb.pdf}}
\hlink[js:\DDDcontext{malte}.clockWise();]      {\includegraphics[height=1.44em]{boutonc.pdf}}
\hlink[js:\DDDcontext{malte}.scaleSpeed(1/1.1);]{\includegraphics[height=1.44em]{boutond.pdf}}
\hlink[js:\DDDcontext{malte}.origSpeed();]      {\includegraphics[height=1.44em]{boutone.pdf}}
\hlink[js:\DDDcontext{malte}.scaleSpeed(1.1);]  {\includegraphics[height=1.44em]{boutonf.pdf}}


\end{document}
